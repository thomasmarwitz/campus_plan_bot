
\documentclass{article}
\usepackage[a4paper, margin=1in]{geometry}
\usepackage{longtable}
\usepackage{amssymb}
\usepackage{graphicx}
\title{Chatbot Evaluation Report Snippets}
\author{Generated by evaluation framework}
\date{\today}
\begin{document}
\maketitle
\section*{Category: Opening Hours}
\subsection*{Example: Failure (Score = 0.00)}
\begin{longtable}{p{0.2\textwidth} p{0.8\textwidth}}
\textbf{Input} & Zu welchen Zeiten hat die Ambulanz geöffnet? \\
\textbf{Expected Output} & Das Gebäude hat an 24/7 geöffnet. \\
\textbf{Actual Output} & Ich denke, du fragst nach den Öffnungszeiten der Ambulanz. Leider habe ich keine Informationen über eine Ambulanz auf dem KIT-Campus gefunden. Die gelieferten Dokumente beziehen sich auf Institute, Einrichtungen und Gebäude, aber keine Ambulanz. Wenn du möglicherweise eine andere Frage hast, stehe ich gerne zur Verfügung. \\
\textbf{Judge Reason} & The output does not convey the required information about the 24/7 opening hours of the ambulance and instead provides an excuse about lacking data. \\
\textbf{Scores} & FScore: 0.61, Precision: 0.57, Recall: 0.65 \\
\end{longtable}
\subsection*{Example: Partial Success (Score = 0.75)}
\begin{longtable}{p{0.2\textwidth} p{0.8\textwidth}}
\textbf{Input} & Wann ist Gebäude 01.13 zugänglich? \\
\textbf{Expected Output} & Das Gebäude hat an Mo-Fr 11:00-14:00, 14:00-14:30 open "Spätausgabe an Linie 1"; Sa,Su,PH off geöffnet. \\
\textbf{Actual Output} & Das Gebäude 01.13 ist montags bis freitags von 11:00 bis 14:00 Uhr und zusätzlich von 14:00 bis 14:30 Uhr geöffnet, wenn es eine Spätausgabe an die Linie 1 gibt. Am Samstag und Sonntag ist es geschlossen. \\
\textbf{Judge Reason} & The output conveys the correct meaning about the building's accessibility but does not include the information about being closed on public holidays (PH) as present in the expected output. \\
\textbf{Scores} & FScore: 0.76, Precision: 0.76, Recall: 0.77 \\
\end{longtable}
\subsection*{Example: Success (Score = 1.00)}
\begin{longtable}{p{0.2\textwidth} p{0.8\textwidth}}
\textbf{Input} & Wann ist das Institut für Bio- und Lebensmitteltechnik, Bereich IV: Molekulare Aufarbeitung von Bioprodukten zugänglich? \\
\textbf{Expected Output} & Das Gebäude hat an Mo-Fr 07:00-18:00; Sa,Su,PH off geöffnet. \\
\textbf{Actual Output} & Das Institut für Bio- und Lebensmitteltechnik, Bereich IV: Molekulare Aufarbeitung von Bioprodukten ist von Montag bis Freitag von 07:00 bis 18:00 Uhr geöffnet. \\
\textbf{Judge Reason} & The output correctly conveys the opening hours (Monday to Friday, 07:00-18:00) as in the expected output without adding unnecessary information or excuses. \\
\textbf{Scores} & FScore: 0.70, Precision: 0.68, Recall: 0.73 \\
\end{longtable}
\section*{Category: Building Location}
\subsection*{Example: Failure (Score = 0.00)}
\begin{longtable}{p{0.2\textwidth} p{0.8\textwidth}}
\textbf{Input} & Wo liegt das Institutsgebäude Nachrichtentechnik? \\
\textbf{Expected Output} & Das Gebäude hat die Adresse Institutsgebäude Nachrichtentechnik, 5, Engesserstraße, Innenstadt-Ost Nordöstlicher Teil, Innenstadt-Ost, Karlsruhe, Baden-Württemberg, 76131, Deutschland. \\
\textbf{Actual Output} & Das Institut für Nachrichtentechnik liegt in der Nordweststadt, Hertzstraße 16. \\
\textbf{Judge Reason} & The output provides a different address than the expected output, which is incorrect. \\
\textbf{Scores} & FScore: 0.71, Precision: 0.74, Recall: 0.68 \\
\end{longtable}
\subsection*{Example: Partial Success (Score = 0.50)}
\begin{longtable}{p{0.2\textwidth} p{0.8\textwidth}}
\textbf{Input} & An welcher Stelle befindet sich Gebäude 239? \\
\textbf{Expected Output} & Das Gebäude hat die Adresse Eggensteiner Straße, Leopoldshafen, Eggenstein-Leopoldshafen, Landkreis Karlsruhe, Baden-Württemberg, 76344, Deutschland. \\
\textbf{Actual Output} & Das Gebäude mit der Nummer 239 befindet sich in der Eggensteiner Straße im Stadtviertel Leopoldshafen. Es handelt sich um einen Parkplatz. \\
\textbf{Judge Reason} & The output conveys the location of Gebäude 239 as Eggensteiner Straße in Leopoldshafen but adds extra information about it being a parking lot, which is not present in the expected output. \\
\textbf{Scores} & FScore: 0.73, Precision: 0.74, Recall: 0.73 \\
\end{longtable}
\subsection*{Example: Success (Score = 1.00)}
\begin{longtable}{p{0.2\textwidth} p{0.8\textwidth}}
\textbf{Input} & Wie lautet die Adresse von Gebäude 523? \\
\textbf{Expected Output} & Das Gebäude hat die Adresse Campus Nord, Leopoldstraße, Kirchfeld, Neureut, Karlsruhe, Dettenheim, Baden-Württemberg, 76131, Deutschland. \\
\textbf{Actual Output} & Die Adresse von Gebäude 523 lautet Campus Nord, Leopoldstraße, 76131 Karlsruhe, Neureut. \\
\textbf{Judge Reason} & The output conveys the same address information as the expected output, though the order and formatting differ slightly. \\
\textbf{Scores} & FScore: 0.83, Precision: 0.86, Recall: 0.81 \\
\end{longtable}
\section*{Category: Open Now}
\subsection*{Example: Failure (Score = 0.00)}
\begin{longtable}{p{0.2\textwidth} p{0.8\textwidth}}
\textbf{Input} & Kann man gerade ins InformatiKOM 1? \\
\textbf{Expected Output} & Nein, das Gebäude ist momentan geschlossen. \\
\textbf{Actual Output} & Ja, das InformatiKOM 1 ist gerade zugänglich. Es befindet sich am Adenauerring 12 in der Oststadt und hat die folgenden Öffnungszeiten: Montag bis Freitag von 09:00 bis 19:00 Uhr und 24/7 mit KIT-Card. \\
\textbf{Judge Reason} & The output contradicts the expected answer and provides unnecessary information instead of confirming the building's current status. \\
\textbf{Scores} & FScore: 0.66, Precision: 0.59, Recall: 0.74 \\
\end{longtable}
\subsection*{Example: Partial Success (Score = 0.50)}
\begin{longtable}{p{0.2\textwidth} p{0.8\textwidth}}
\textbf{Input} & Steht Gebäude 01.90 momentan offen? \\
\textbf{Expected Output} & Ja, das Gebäude ist momentan geöffnet. \\
\textbf{Actual Output} & Momentan ist das Internet Café Aso (Gebäude 01.90) offen, da es Montag bis Freitag von 09:00 bis 22:00 Uhr und Samstag von 10:00 bis 22:00 Uhr geöffnet hat. \\
\textbf{Judge Reason} & The output conveys the meaning that Gebäude 01.90 is currently open, but adds unnecessary details not present in the expected output. The information is correct, but the extra explanation may be considered an excuse rather than a concise response. \\
\textbf{Scores} & FScore: 0.68, Precision: 0.60, Recall: 0.77 \\
\end{longtable}
\subsection*{Example: Success (Score = 1.00)}
\begin{longtable}{p{0.2\textwidth} p{0.8\textwidth}}
\textbf{Input} & Hat die Fachschaft Informatik momentan auf? \\
\textbf{Expected Output} & Ja, das Gebäude ist momentan geöffnet. \\
\textbf{Actual Output} & Die Fachschaft Informatik ist momentan auf. Sie befindet sich im Informatikbau am Am Fasanengarten 5 und hat folgende Öffnungszeiten: Montag bis Freitag von 07:00 bis 22:30 Uhr und Samstag von 07:00 bis 15:00 Uhr. Sie ist auch rollstuhlgerechtig. \\
\textbf{Judge Reason} & The output correctly conveys that the Fachschaft Informatik is currently open, which matches the meaning of the expected output. \\
\textbf{Scores} & FScore: 0.65, Precision: 0.58, Recall: 0.74 \\
\end{longtable}
\section*{Category: Open Until}
\subsection*{Example: Failure (Score = 0.00)}
\begin{longtable}{p{0.2\textwidth} p{0.8\textwidth}}
\textbf{Input} & Bis wann kann ich das IAM-ET Gebäude noch betreten? \\
\textbf{Expected Output} & Das Gebäude schließt in 8 Stunden und 46 Minuten. \\
\textbf{Actual Output} & Ich kann dir nicht genau sagen, bis wann das IAM-ET Gebäude noch betreten werden kann, da ich keine Informationen über dieses Gebäude in meiner Datenbank habe. Es ist jedoch möglich, dass das IAM-ET Gebäude mit dem Oberen Hörsaal Maschinenbau identisch ist, der in der Nähe des Alten Maschinenbaugebäudes liegt. Das Alte Maschinenbaugebäude ist jedoch nicht das IAM-ET Gebäude. Wenn du nach Informationen über das IAM-ET Gebäude suchst, empfehle ich dir, die offizielle Website des KIT zu besuchen oder das Gebäudepersonal zu kontaktieren. \\
\textbf{Judge Reason} & The output does not convey the required information about the IAM-ET building's closing time and instead provides excuses and irrelevant details. \\
\textbf{Scores} & FScore: 0.61, Precision: 0.58, Recall: 0.65 \\
\end{longtable}
\subsection*{Example: Partial Success (Score = 0.50)}
\begin{longtable}{p{0.2\textwidth} p{0.8\textwidth}}
\textbf{Input} & Wann schließt die Bibliothek am Campus Nord? \\
\textbf{Expected Output} & Das Gebäude schließt in 0 Stunden und 54 Minuten. \\
\textbf{Actual Output} & Die Bibliothek am Campus Nord schließt am heutigen Tag um 15:00 Uhr. \\
\textbf{Judge Reason} & The output provides a specific closing time (15:00 Uhr) but does not match the expected phrasing or convey the same meaning as 'Das Gebäude schließt in 0 Stunden und 54 Minuten.' \\
\textbf{Scores} & FScore: 0.78, Precision: 0.77, Recall: 0.79 \\
\end{longtable}
\subsection*{Example: Success (Score = 1.00)}
\begin{longtable}{p{0.2\textwidth} p{0.8\textwidth}}
\textbf{Input} & Bis wann hat Gebäude 403 heute auf? \\
\textbf{Expected Output} & Für dieses Gebäude kenne ich leider keine Öffnungszeiten. \\
\textbf{Actual Output} & Ich kann keine Öffnungszeiten für Gebäude 403 finden. Es ist möglich, dass es sich um ein nicht existierendes Gebäude handelt oder die Informationen nicht verfügbar sind. \\
\textbf{Judge Reason} & The output conveys the same meaning as the expected output by indicating that the building's opening hours are unknown \\
\textbf{Scores} & FScore: 0.74, Precision: 0.68, Recall: 0.82 \\
\end{longtable}
\section*{Category: Closed Until}
\subsection*{Example: Failure (Score = 0.00)}
\begin{longtable}{p{0.2\textwidth} p{0.8\textwidth}}
\textbf{Input} & Ab wann ist der ZEISS Makerspace @ KIT wieder zugänglich? \\
\textbf{Expected Output} & Das Gebäude öffnet in 0 Tagen, 9 Stunden und 11 Minuten. \\
\textbf{Actual Output} & Ich kann keine Informationen über den ZEISS Makerspace @ KIT finden. Es scheint, dass der ZEISS Makerspace nicht in meiner Datenbank enthalten ist. Könnten Sie mir mehr über den Makerspace sagen, damit ich Ihnen besser helfen kann? \\
\textbf{Judge Reason} & The output does not convey the required information about the ZEISS Makerspace accessibility and instead asks for more details. \\
\textbf{Scores} & FScore: 0.57, Precision: 0.54, Recall: 0.59 \\
\end{longtable}
\subsection*{Example: Partial Success (Score = 0.50)}
\begin{longtable}{p{0.2\textwidth} p{0.8\textwidth}}
\textbf{Input} & Wann macht Gebäude 513 auf? \\
\textbf{Expected Output} & Für dieses Gebäude kenne ich leider keine Öffnungszeiten. \\
\textbf{Actual Output} & Das Gebäude mit der Nummer 513 scheint nicht öffentlich zugänglich zu sein, da es sich um die Adresse der Kerntechnischen Entsorgung Karlsruhe GmbH handelt. Es ist also wahrscheinlich nicht geöffnet. \\
\textbf{Judge Reason} & The output conveys the information that building 513 is likely not open, which is consistent with the expected message that no opening times are known. However, it goes beyond the expected output by providing additional context that might not be necessary. \\
\textbf{Scores} & FScore: 0.67, Precision: 0.63, Recall: 0.71 \\
\end{longtable}
\subsection*{Example: Success (Score = 1.00)}
\begin{longtable}{p{0.2\textwidth} p{0.8\textwidth}}
\textbf{Input} & Ab wann kann man wieder in den Informatikbau? \\
\textbf{Expected Output} & Das Gebäude ist grade geöffnet. \\
\textbf{Actual Output} & Der Informatikbau ist derzeit von Montag bis Freitag von 07:00 bis 22:30 und am Samstag von 07:00 bis 15:00 geöffnet. Sonntags und an Feiertagen ist er geschlossen. Er ist rollstuhlgeeignet. \\
\textbf{Judge Reason} & The output provides detailed and correct information about the opening hours of the Informatikbau, which aligns with the expected output's statement that the building is currently open. The information is conveyed clearly without listing excuses. \\
\textbf{Scores} & FScore: 0.69, Precision: 0.61, Recall: 0.81 \\
\end{longtable}
\section*{Category: Navigation Link}
\subsection*{Example: Failure (Score = 0.00)}
\begin{longtable}{p{0.2\textwidth} p{0.8\textwidth}}
\textbf{Input} & Zeige den Weg zur Westhochschule \\
\textbf{Expected Output} & Mit diesem Link kannst du dich zu dem Gebäude führen lassen: https://www.google.com/maps/dir/?api=1\textbackslash{}&destination=49.02041,8.36768. \\
\textbf{Actual Output} & Um zur Westhochschule zu gelangen, folge bitte den folgenden Anweisungen:\textbackslash{}n\textbackslash{}nDie Westhochschule liegt im Stadtteil Innenstadt-Ost. Von hier aus kannst du entweder zu Fuß oder mit dem Bus dorthin gelangen.\textbackslash{}n\textbackslash{}nDie Adresse der Westhochschule lautet: Wilhelm-Nusselt-Weg, 76131 Karlsruhe.\textbackslash{}n\textbackslash{}nWenn du zu Fuß gehst, folge der Wilhelm-Nusselt-Straße in Richtung Osten. Die Westhochschule befindet sich auf der rechten Seite.\textbackslash{}n\textbackslash{}nWenn du den Bus nimmst, steige an der Haltestelle "Innenstadt-Ost" aus und gehe von dort aus in Richtung Wilhelm-Nusselt-Weg. Die Fahrt dauert etwa 5-10 Minuten.\textbackslash{}n\textbackslash{}nIch hoffe, das hilft dir! \\
\textbf{Judge Reason} & The output does not provide the required link to Google Maps for directions to the Westhochschule. \\
\textbf{Scores} & FScore: 0.58, Precision: 0.57, Recall: 0.58 \\
\end{longtable}
\subsection*{Example: Partial Success (Score = 0.50)}
\begin{longtable}{p{0.2\textwidth} p{0.8\textwidth}}
\textbf{Input} & Route zum Gebäude 30.79 \\
\textbf{Expected Output} & Mit diesem Link kannst du dich zu dem Gebäude führen lassen: https://www.google.com/maps/dir/?api=1\textbackslash{}&destination=49.01330,8.41711. \\
\textbf{Actual Output} & Das Gebäude mit der Identifikator "30.79" ist das RoofKIT. Die genauen Koordinaten sind 49.01330, 8.41711. Wenn du dich an diesem Punkt befindest, kannst du nach Norden gehen, um das Gebäude zu erreichen. \\
\textbf{Judge Reason} & The output provides information about the building and its coordinates, but it does not include the Google Maps link as specified in the expected output. \\
\textbf{Scores} & FScore: 0.67, Precision: 0.68, Recall: 0.66 \\
\end{longtable}
\section*{Category: Wheelchair Accessible}
\subsection*{Example: Failure (Score = 0.00)}
\begin{longtable}{p{0.2\textwidth} p{0.8\textwidth}}
\textbf{Input} & Gibt es Rollstuhlzugang im Gebäude 210? \\
\textbf{Expected Output} & Ja, das Gebäude ist rollstuhlgerecht. \\
\textbf{Actual Output} & Es tut mir leid, aber ich konnte keine Informationen über den Rollstuhlzugang im Gebäude 210 finden. Allerdings habe ich ein Dokument gefunden, das den Rollstuhlgerechtigkeitsstatus für ein Gebäude mit der Nummer 215 als "yes" (ja) besagt. Es ist möglich, dass dies das Gebäude ist, nach dem du gefragt hast. \\
\textbf{Judge Reason} & The output does not convey the correct information about Building 210 being wheelchair accessible. Instead, it expresses inability to find information and references Building 215, which is irrelevant. \\
\textbf{Scores} & FScore: 0.68, Precision: 0.63, Recall: 0.74 \\
\end{longtable}
\subsection*{Example: Partial Success (Score = 0.80)}
\begin{longtable}{p{0.2\textwidth} p{0.8\textwidth}}
\textbf{Input} & Ist das Institut für Baustatik mit Rollstuhl erreichbar? \\
\textbf{Expected Output} & Ja, das Gebäude ist rollstuhlgerecht. \\
\textbf{Actual Output} & Das Institut für Baustatik ist rollstuhlgerechtig. \\
\textbf{Judge Reason} & The output conveys the same information as the expected output but contains a grammatical error ('rollstuhlgerechtig' instead of 'rollstuhlgerecht') \\
\textbf{Scores} & FScore: 0.82, Precision: 0.81, Recall: 0.83 \\
\end{longtable}
\subsection*{Example: Success (Score = 1.00)}
\begin{longtable}{p{0.2\textwidth} p{0.8\textwidth}}
\textbf{Input} & Hat das Institut für Photogrammetrie und Fernerkundung einen barrierefreien Zugang? \\
\textbf{Expected Output} & Ja, das Gebäude ist rollstuhlgerecht. Es gibt Rollstuhlrampe auf der Rückseite des Gebäudes. Aufzug befindet sich links. Behindertengerechten Toiletteim UG nähe Rampe. \\
\textbf{Actual Output} & Das Institut für Photogrammetrie und Fernerkundung (IPF) hat einen barrierefreien Zugang. Es gibt eine Rollstuhlrampen auf der Rückseite des Gebäudes und einen Aufzug, der links liegt. Außerdem gibt es eine Behindertentoilette im Untergeschoss in der Nähe der Rampe. \\
\textbf{Judge Reason} & The output conveys the same information as the expected output, including the availability of a wheelchair ramp, elevator location, and accessible restroom. \\
\textbf{Scores} & FScore: 0.75, Precision: 0.73, Recall: 0.76 \\
\end{longtable}
\section*{Category: Open Website}
\subsection*{Example: Failure (Score = 0.00)}
\begin{longtable}{p{0.2\textwidth} p{0.8\textwidth}}
\textbf{Input} & Zeige mir die Homepage des Instituts für Wasser und Gewässerentwicklung - Wasserwirtschaft und Kulturtechnik \\
\textbf{Expected Output} & Ich habe die Webseite für dich in deinem Browser geöffnet. \\
\textbf{Actual Output} & Das Institut für Wasser und Gewässerentwicklung - Wasserwirtschaft und Kulturtechnik ist unter folgender Adresse zu finden: Engesserstraße 22, 76131 Karlsruhe. Die Homepage des Instituts ist unter https://www.iwu.kit.edu/ erreichbar. \\
\textbf{Judge Reason} & The output conveys the correct information about the institute's homepage URL, while the expected output states that the bot has opened the webpage. The output does not match the expected meaning. \\
\textbf{Scores} & FScore: 0.59, Precision: 0.56, Recall: 0.63 \\
\end{longtable}
\subsection*{Example: Partial Success (Score = 0.50)}
\begin{longtable}{p{0.2\textwidth} p{0.8\textwidth}}
\textbf{Input} & ["Navigate to the HOC building's website"] \\
\textbf{Expected Output} & Ich habe die Webseite für dich in deinem Browser geöffnet. \\
\textbf{Actual Output} & Die Website des HOC-Buildings ist http://www.hoc.kit.edu/. \\
\textbf{Judge Reason} & The output provides the URL of the HOC building's website, while the expected output mentions opening the website in the browser. The meaning of performing an action related to the website is conveyed but not in the same way. \\
\textbf{Scores} & FScore: 0.60, Precision: 0.57, Recall: 0.64 \\
\end{longtable}
\subsection*{Example: Success (Score = 1.00)}
\begin{longtable}{p{0.2\textwidth} p{0.8\textwidth}}
\textbf{Input} & Webadresse des Studierendenservice gesucht \\
\textbf{Expected Output} & Ich habe die Webseite für dich in deinem Browser geöffnet. \\
\textbf{Actual Output} & Das Webadresse des Studierendenservice ist http://www.sle.kit.edu. \\
\textbf{Judge Reason} & The output provides the web address, while the expected output mentions opening the webpage in the browser. Both convey the necessary information about the Studierendenservice website. \\
\textbf{Scores} & FScore: 0.60, Precision: 0.57, Recall: 0.64 \\
\end{longtable}
\end{document}
